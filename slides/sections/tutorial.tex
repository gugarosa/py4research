\section{Aplicação prática em Python}
\label{s.python_application}

\begin{frame}{Aplicação prática em Python}
	\justify 
	\begin{itemize}
		\item<1> O principal objetivo deste \textbf{tutorial} é construir uma \textbf{aplicação prática} em \textbf{Python};
		\\~\\
		\item<2> Aplicação será um \textbf{pacote/biblioteca}, hospedado no GitHub e compartilhado no repositório oficial PyPI;
		\\~\\
		\item<3> Pacote/biblioteca será composto por arquivos básicos para \textbf{instalação}, \textbf{dependências}, \textbf{documentação} e \textbf{testes unitários};
		\\~\\
		\item<4> Por fim, serão apresentados alguns exemplos de como \textbf{melhorar} a \textbf{divulgação} e \textbf{atrair} o uso da \textbf{comunidade}.
	\end{itemize}
\end{frame}

\subsection{Criação de repositório}
\label{ss.repository_creation}

\begin{frame}{Criação de repositório}
	\begin{figure}
		\centering
		\includegraphics[scale=0.125]{figs/video_play.png}
	\end{figure}
\end{frame}

\subsection{Implementação}
\label{ss.implementation}

\begin{frame}{Implementação}
	\begin{alertblock}{}
		\centering
		\url{https://github.com/gugarosa/py4research}
	\end{alertblock}
\end{frame}

\begin{frame}{}
	\justify
	\begin{itemize}
		\item<1> \texttt{docs/};
		\\~\\
		\item<2> \texttt{examples/};
		\\~\\
		\item<3> \texttt{py4research/};
		\\~\\
		\item<4> \texttt{tests/};
		\\~\\
		\item<5> \texttt{requirements.txt e setup.py}.
	\end{itemize}
\end{frame}

\subsection{Submissão ao PyPI}
\label{ss.pypi_submission}

\begin{frame}{Submissão ao PyPI}
	\begin{figure}
		\centering
		\includegraphics[scale=0.125]{figs/video_play.png}
	\end{figure}
\end{frame}

\subsection{Locais de divulgação}
\label{ss.places_sharing}

\begin{frame}{Locais de divulgação}
	\justify
	\begin{itemize}
		\item \textbf{Academia}: Pré-publicações, conferências, \emph{workshops}, simpósios, dentre outros;
		\\~\\
		\item \textbf{Ambiente profissional}: Colegas de laboratório/trabalho, colaboradores de pesquisa e professores;
		\\~\\
		\item \textbf{Blogs}: Medium (Towards Data Science, freeCodeCamp.org e 3 min read);
		\\~\\
		\item \textbf{Fóruns}: Reddit, Stack Overflow e Hacker News;
		\\~\\
		\item \textbf{Redes Sociais}: LinkedIn, Twitter e YouTube.
	\end{itemize}
\end{frame}